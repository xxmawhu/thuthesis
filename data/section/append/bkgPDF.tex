\section{Background Shape}
\label{Sec: appendix background shape}
In the procedure of set UL, the signal shape is modeled by the MC
shape, using the \textbf{RooKeysPdf}. Still the uncertainties of
background shape will affect the UL. To estimate such uncertainty, the
different shapes of background are obtained by changing the smooth
parameter "rho" from 1.0 to 4.0. The change of background shape is
shown in Fig. \ref{fig:rho}
\begin{figure}[htbp]
    \begin{center}
    \mbox{
        \begin{overpic}[width = 0.7 \linewidth]{./section/append/fig/rho.eps}
        \put(25,70) { }
        \end{overpic}
    }
    \end{center}
    \caption{Change the smooth parameter. The black points with error
        bars are the MC sample. The red solid line is the
    nominal background shape, while the dotted line denote possible
candidates.}
    \label{fig:rho}
\end{figure}

To test the goodness of those PDFs, we do I/O check for the "rho"
value 1.0, 2.0 and 4.0. The pull distributions are shown in Fig. 
\ref{fig:pull_rho}
\begin{figure}[htbp]
    \begin{center}
    \mbox{
        \begin{overpic}[width = 0.3
            \linewidth]{/besfs/groups/jpsi/jpsigroup/user/maxx/4180/PPbarEnu_newPID/IOcheck/test/rho_1/fit/Pull_fit.eps}
            \put(22,70) { $\rho=1.0$ }
        \end{overpic}
        \begin{overpic}[width = 0.3
            \linewidth]{/besfs/groups/jpsi/jpsigroup/user/maxx/4180/PPbarEnu_newPID/IOcheck/test/rho_2/fit/Pull_fit.eps}
            \put(22,70) { $\rho=2.0$ }
        \end{overpic}
        \begin{overpic}[width = 0.3
            \linewidth]{/besfs/groups/jpsi/jpsigroup/user/maxx/4180/PPbarEnu_newPID/IOcheck/test/rho_4/fit/Pull_fit.eps}
            \put(22,70) { $\rho=4.0$ }
        \end{overpic}
    }
    \end{center}
    \caption{}
    \label{fig:pull_rho}
\end{figure}



