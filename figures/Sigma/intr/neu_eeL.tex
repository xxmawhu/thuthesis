\documentclass[varwidth]{standalone}

\usepackage{tikz}%
\tikzstyle arrowstyle=[scale=1.5]

\tikefeynmanet{%
    myvertex/.style {%
    shape =circle,
    draw=blue,
    fill=blue,
  }
}

\usepackage[compat=1.1.0]{tikz-feynman}



\begin{document}

\begin{tikzpicture}
  \begin{feynman}
    \vertex (i1) {\(\Sigma^{0}\)};
    \vertex[right=3cm of i1] (a);
    \vertex[right=3cm of a] (b);
    \vertex[right=3cm of b] (f) {\(\Lambda\)};

    \vertex[below=2cm of a] (d1);
    \vertex[right=3cm of d1] (d2);
    
    \vertex[below=1.1cm of d1] (vir);
    \vertex[below=-0.4cm of d1] (vir2);
    \vertex[right=6cm of vir] (g1) {\(e^{-}\)};
    \vertex[right=6cm of vir2] (g2) {\(e^{+}\)};
    \diagram* {
        (i1) --[baryon,very thick] (a) --[ultra thick] (b) --[baryon,very thick] (f),
        (a) --[mphoton,very thick] (d1) --[very thick] (d2) -- [mphoton,very thick](b),
        (d2)--[electron,very thick] (g2) ,
        (g1)--[electron,very thick](d1),
    };

   % \draw [decoration={brace}, decorate] (b1.south west) -- (a1.north west)
%          node [pos=0.5, left] {\(B^{0}\)};
   % \draw [decoration={brace}, decorate] (c3.north east) -- (c1.south east)
    %      node [pos=0.5, right] {\(\pi^{-}\)};
    %\draw [decoration={brace}, decorate] (a6.north east) -- (b5.south east)
%          node [pos=0.5, right] {\(\pi^{+}\)};
  \end{feynman}
\end{tikzpicture}

\end{document}

