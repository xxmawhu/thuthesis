\chapter{总结和展望}%
\label{sec:sunmmary}
% 本文的研究内容小结
近年来北京正负电子对撞机实验采集了大量的粲介子样本和世界上最大的$J/\psi$样本,
提供了研究粲介子和粲偶素的理想场所。搭乘这个顺风车,我们得以开展对重子的初步研究。
寻找重子对的来源仍任重道远,在$2.78 {\rm fb}^{-1}$的$D_{s}$的数据样本中,我们在
$90\%$的置信度下否定了$D_{s}^{+} \to p\bar{p} e^{+} \nu_{e}$的存在,同时我们宣布了在
含粲介子$D_{s}^{+} \to p \bar{n}$至今仍是中唯一的一个能够产生重子的过程。
在BESIII产生的$c\bar{c}$共振态中,$J/\psi$仍是正反重子的最大来源,同时提供了大量的
$\eta_{c}$样本,这提供了研究$\eta_{c}$到正反重子对衰变过程的良机,一方面$eta_{c}$的
分支比测量存在很多空白,尚有$36\%$的衰变过程仍然未知。在$3.097~{\rm GeV}$能量点
我们有选择性了对$J/\psi \to \Sigma^{0} \bar{\Sigma^{0}}$进行了深入细致的研究,重点测量
了衰变参数$\alpha_{J/psi}$及形状因子$G_{E(M)}$之间的相位差,发现SU(3)对称性发生了大的破缺,
非但不同重子对的$\alpha_{J/\psi}$的值存在差异,甚至符号都不尽相同。同时SU(2)对称性也有破缺,
$\Sigma^{0}\bar{\Sigma}^{0}$的两个形状因子的相位差与$\Sigma^{+} \bar{\Sigma}^{-}$的存在较大
差异,目前尚缺少可靠的理论解释。
充分利用当前的样本,我们着重研究了$\Sigma^{0}$的一个重要的三体衰变过程
$\Sigma^{0} \to \Lambda e^{+} e^{-}$,并初步给出了分支比,在$e^{+}e^{-}$的质量谱上
没有看到显著的$X(17)$的贡献,这将有助于加深对$X(17)$的理解。
% BESIII实验现状
在可见的预期内,BESSIII合作组将在重子物理方面继续做出卓越的贡献,特别是对量子关联的超子对的研究。
对这种特殊的量子关联现象的一系列研究将直接检验量子力学,其中的代表便是检验贝尔不等式,
这势必加深我们对量子力学的基础的理解。合作组在不远的未来将对重子八重态成对产生进行全面的研究,这将对
我们理解超子的结构函数,超子和$c\bar{c}$共振态的相互作用,超子的衰变形状提供更为广泛的实验数据。
在世界上最大的$J/\psi$样本的基础上,在BESIII对撞机低本底的优势下,我们在寻找超子的稀有衰变也独树一帜。
包括超子的含中子衰变的衰变常数测量,超子的辐射衰变研究,超子的达利兹衰变测量上。
% 未来的方向 
在未来的,多个实验组将在超子物理上展开角着,包括PANDAS,super-tau-charm,LHCb等。世界上的超子对的
样本大小将在提高至少一个量级,更精确的测量将得以展开,更稀有的衰变可能被陆续观测到。
与此同时实验家门将面更大的挑战。包括大统计量下的技术难题,精确测量特别的寻找CP破坏下的系统误差的控制。
在高精度测量下,任何忽略潜在的问题都可能带来致命的测量偏差。比如:
在正反重子对产生机制里,交换双光子过程常常被忽略;实验分析上,初态及末态辐射的贡献总是被忽略;
超子和物质的相互与物质的相互作用常常不被重视,超子磁矩和物质相互作用后自旋是否发生改变尚有待回答;



