\section{Fitting}
The systematic uncertainty associated with fitting are summarized in
\ref{uncertainty about fitting}. 
    
For uncertainty associated with fit range, we move the range of
mass of $D_{s}^{+}$ mesons about 5 MeV both left and right, then
choose the biggest difference as the uncertainty. 

\begin{figure}[htbp]
    \label{Systematic uncertainty}
    \mbox{
        \begin{overpic}[width=6cm]{section/append/fig/Ds401_data.eps}
            \put(35,70) {{(1)}}
        \end{overpic}
        \begin{overpic}[width=6cm]{section/append/fig/Ds401_mvrange.eps}
            \put(35,70) { {(1')}}
        \end{overpic}
    }
    \mbox{
        \begin{overpic}[width=6cm]{section/append/fig/Ds400_data.eps}
            \put(35,70) {{(2)}}
        \end{overpic}
        \begin{overpic}[width=6cm]{section/append/fig/Ds400_mvrange.eps}
            \put(35,70) { {(2')}}
        \end{overpic}
    }
    \mbox{
        \begin{overpic}[width=6cm]{section/append/fig/Ds406_data.eps}
            \put(35,70) {{(3)}}
        \end{overpic}
        \begin{overpic}[width=6cm]{section/append/fig/Ds406_mvrange.eps}
            \put(35,70) { {(3')}}
        \end{overpic}
    }
    \caption{Move the range of $M(\Ds)$. left: before move. right: after move. (1) $K^{+}K^{-} \pi ^{+}$ (2) $K_{S} K^{+}$ (3) $ K_{S} K^{+} \pi^{-}  \pi ^{-}$ }
\end{figure}


\begin{figure}[htbp]
    \label{change background}
    \mbox{
        \begin{overpic}[width=6cm]{section/append/fig/Ds401_data.eps}
            \put(35,70) {{(1)}}
        \end{overpic}
        \begin{overpic}[width=6cm]{section/append/fig/Ds401_changebk.eps}
            \put(35,70) { {(1')}}
        \end{overpic}
    }
    \mbox{
        \begin{overpic}[width=6cm]{section/append/fig/Ds400_data.eps}
            \put(35,70) {{(2)}}
        \end{overpic}
        \begin{overpic}[width=6cm]{section/append/fig/Ds400_changebk.eps}
            \put(35,70) { {(2')}}
        \end{overpic}
    }
    \mbox{
        \begin{overpic}[width=6cm]{section/append/fig/Ds406_data.eps}
            \put(35,70) {{(3)}}
        \end{overpic}
        \begin{overpic}[width=6cm]{section/append/fig/Ds406_changebk.eps}
            \put(35,70) { {(3')}}
        \end{overpic}
    }
    \caption{Change background. left: before change. right: after change. (1) $K^{+}K^{-} \pi ^{+}$ (2) $K_{S} K^{+}$ (3) $ K_{S} K^{+} \pi^{-}  \pi ^{-}$ }
\end{figure}

To estimate the uncertainty caused by the shape of signal and
background. We change the shape of background from 2nd Chebychev to 3rd
Chebychev and take the change of yield as uncertainty.  

And we change the way of the shape of the signal, one is getting the
shape of the signal sort the tool of \textbf{RooKeysPdf}, another is
that assume the shape is the sum of two Gaussian Function, and obtain
the parameters from MC sample.  Then we take the change of yield as
uncertainty.

\begin{table}[htbp]
    \caption{Systematic uncertainty }
    \label{uncertainty about fitting}
    \center{
        \begin{tabular}{lccc} 
            \hline \hline
            item & fit range & shape of background & shape of signal \\ \hline
            uncertainty & 0.3\%  & 0.2\%  & 0.5 \%  \\ 
            \hline \hline
        \end{tabular} 
    }
\end{table}

