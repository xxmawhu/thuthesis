% ====================================================
%   Copyright (C)2019 All rights reserved.
%
%   Author        : Xin-Xin Ma
%   Email         : xxmawhu@163.com
%   File Name     : appendix_ds.tex
%   Last Modified : 2019-10-25 18:34
%   Describe      :
%
% ====================================================%

\chapter{关于$D_{s}^{+} \to p \bar{p} e^{+} \nu_{e}$的分析}
\section{超子弱衰变振幅}
本小节讨论半自旋超子的弱衰变振幅。通常基态半自旋超子会通过弱相互作用
衰变到更轻的超子并发射出一个赝标量粒子,例如:
\begin{equation}
    \begin{aligned}
        \label{eq:}
        \Lambda &\to p \pi^{-} \\ \notag
        \Xi &\to \Lambda \pi \\ \notag
    \end{aligned}
\end{equation}
因此本小节特意讨论这一最为常见的衰变,即 $1/2^{+} \to 1/2^{+} 0^{-}$。
考虑到CP守恒和洛伦茨不变形,跃迁振幅可以一般性的写出
\begin{equation}
    M = G_{F} m^{2}_{\pi} \cdot \bar{B}_{f} \left(A-B\gamma_{5}\right) B_{i}
\end{equation}
式中$B_{i}$,$B_{f}$分别是初末态重子旋量,$G_{F}$为费米常数,$m_{\pi}$是$\pi$
的质量,$A$,$B$是耦合常数。

稍微化简可得:
\begin{equation}
    \begin{aligned}
        \label{eq:}
        |M|^{2}& =   \frac{4}{m_{\Lambda}}
    m_{\pi}^4 G_F^2 \Bigg(A^* m_{\Lambda} \bigg(i B \det
   \left(p_{\Lambda},p_{\Xi},s_{\Lambda},s_{\Xi}\right)+A
   m_{\Xi} s_i.s_f \left(E_{\Lambda}+m_{\Lambda}\right)+A
   E_{\Lambda} m_{\Xi}+A m_{\Lambda} m_{\Xi} \\
        &+B q m_{\Xi} n_{\Lambda }.s_f
        +B q m_{\Xi } n_{\Lambda }.s_i\bigg)
        + B^*m_{\Lambda } \bigg(-i A \det \big(p_{\Lambda },p_{\Xi
   },s_{\Lambda },s_{\Xi }\big)+m_{\Xi } n_{\Lambda }.s_f
   \Big(A q \\ 
        &+2 B \big(E_{\Lambda }-m_{\Lambda }\big)
   n_{\Lambda }.s_i \Big)
        +A q m_{\Xi } n_{\Lambda }.s_i+B
   m_{\Xi } s_i.s_f \left(m_{\Lambda }-E_{\Lambda }\right)+B
   E_{\Lambda } m_{\Xi }-B m_{\Lambda } m_{\Xi
   }\bigg)\Bigg)
    \end{aligned}
\end{equation}


\section{分支比和形状因子}
\section{质子鉴别的系统误差研究}


