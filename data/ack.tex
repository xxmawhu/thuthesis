\begin{acknowledgements}
光阴荏苒,岁月如梭,我的博士生涯迎来最后一幕。回首这五年在高能物理研究
所的成长经历,其中走过不少弯路,也幸运地得到贵人相助。能够克服种种困难,
成功的完成研究课题,不仅仅要靠自身的努力,还要归功于天时地利人和。我进入
高能物理研究所时,刚加入BESIII合作组,就幸运的遇到在4.180GeV取数的良好
时机,才得以开展我的第一个课题。而高能所是国外粒子物理研究的最好平台,这
是独一无二的地利。然而,人的主观能动性是起确定性作用的。离开周围人的热心
帮助,天时地利也黯然无光。在这里谨此向多年来给予我帮助的老师、同学及家
人致以深深的谢意!
首先,感谢我的导师李海波研究员。李老师学识渊博、视野开阔,一直是
我的学习榜样。在科研上李老师以身作则,思维严密、态度严谨,将自己的科
研经验无私地传授于我们,孜孜不倦地教导我们要主动学习、踏踏实实的做研
究、努力认真地完成自己的工作,详细耐心地指导我的工作,加快了我在科研
道路上的步伐。李老师默默地、无怨无悔地、不求回报地付出,他那发自内心
的、 渴望将自己的学生培养成才、 希望我们能够在某一方面有所建树的眼神,
也许一直都是我前进的动力。李老师还时刻关心我们的思想工作,真诚而又恰
到好处地指出我们存在的问题。在我的人生的转折点上,李老师又不厌其烦地
一遍又一遍地为我指引着前进的方向。千言万语无法表达我的仰慕感激之情,
在此谨向李老师致以崇高的谢意!

首先,我要感谢我的博士生导师李海波研究员。李老师学识丰富、思维开
阔而又待人谦和、平易近人。在科研过程中,从对物理内涵的理解到对物理细
节的把握,李老师严肃的科学精神和严谨的治学态度都深深地感染了我,因此
我一直将李老师树立为学习的榜样。在生活中,李老师也给予了我许多关怀和
鼓励。我在博士生涯中所获取的知识和成果,都离不开李老师的指导和帮助。
同时,我要感谢我的良师益友吕晓睿教授。吕老师学识和科研经验丰富而
且待人热情。 在科研过程中经常会碰到各种问题,有时会令人感到一筹莫展。
我有幸与吕老师在同一办公室,得以经常向吕老师请教,每次都使我受益匪
浅。感谢吕老师细致的帮助和指导,使我在科研过程中少走了很多弯路,我的
成长和进步离不开吕老师的帮助。


我还要感谢软件组的袁野研究员。刚来到实验物理中心时,我曾在袁老师
的指导下从事主漂移室的参数调试工作。袁老师给了我很多指导,使我学到了
很多探测器和软件的知识。 此外还要感谢郑阳恒、马海龙、何吉波、朱永生、
伍灵慧、王亮亮、张瑶等老师,与他们的交流同样使我收获良多。
感谢管颖慧师姐和秦小帅、刘凯师兄,在我刚来到实验物理中心的那段时
间里,你们手把手地教我分析技术,使初来乍到的我逐步渐入佳境。 感谢代
建平、刘杰、范荆洲、张瑞、徐庆年和刘佩莲等师兄师姐,和你们的讨论使我
深受启发,不断加深对物理问题的认识。 感谢博士期间的同学和好友:李培
荣、卢宇、周兴玉、陆宇,和你们在一起共度的六年时光会成为人生中美好的
回忆。
感谢我的师门:刘兰雕、刘晓霞、王滨龙、豆正磊、黄晓忠、李慧
静、张宇、周亦雄、杨友华、谷立民、马新鑫、朱傲男、秦佳佳、王丹,和你
们在一起的科研和娱乐给博士生活增添了许多色彩。
最后,我要感谢我的父母,你们多年的支持是我前进道路上不竭的动力!
2017 年 6 月
感谢朝鲁提供的 ucasthesis 模板,它的存在让我的论文写作更加方便以及整洁。

\end{acknowledgements}
