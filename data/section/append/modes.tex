\section{Determine the tag modes}
To test the necessity of the six tag modes, we add tag mode one by one
to the fitting process, from the largest yield to the smallest, from
low-level background to high. The results are listed in Table
\ref{Table: add tag modes}. We conclude that the first three tag modes
are better enough to set upper limit
\cite{Lees:2011qz,Ablikim:2015djc,Ablikim:2017twd}, the result will not
be improved if more tag mode is added, so we decide to only adopt the
three tag modes in our analysis.

\begin{table}[htbp]
    \caption{" $\surd$ "~ means this tag mode is added in the simultaneous fit.}
    \label{Table: add tag modes}
    \center{
        \begin{tabular}[scalor=0.2]{c c c c c c |c}
            \hline    \hline
            $ K^{+} K^{-} \pi^{-}$ & $ K_{S} K^{-}$ & $ K_{S} K^{+} \pi^{-} \pi^{-}$ & $ K^{+} K^{-} \pi^{-} \pi ^{0} $  
            & $ \pi^{+} \pi^{-} \pi^{-}$ &     $  \eta ' \pi^{-} $ &  br(10$ ^{-4}$) \\ \hline

            $ \surd $     &      &      &      &      &     & 2.4    \\     \hline    
            $ \surd $     & $ \surd $     &      &      &     &      & 1.8    \\\hline
            $ \surd $     & $ \surd $     & $ \surd $     &      &      &     & 1.6    \\\hline
            $ \surd $     & $ \surd $     &  $ \surd $    &  $ \surd $    &      &     & 1.6\\\hline
            $ \surd $     & $ \surd $     &  $ \surd $    &  $ \surd $    &  $ \surd $    &     & 1.6\\\hline
            $ \surd $     & $ \surd $     &  $ \surd $    &  $ \surd $    &  $ \surd $    & $ \surd $    & 1.6  \\    \hline  \hline

        \end{tabular}    
    }        
\end{table}

So in the analysis, we decide to only to use the three tag modes.
